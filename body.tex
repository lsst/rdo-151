
\section{Purpose and Scope}\label{purpose-and-scope}

The Vera C.\ Rubin Observatory Publication Policy outlines the approval process for specific published material defined in Section \ref{publications-covered-by-this-policy}.
There are three primary goals for this policy:

\begin{itemize}
\item
  To ensure the accuracy of published information about the requirements, specifications, design, performance, and products of the Rubin Observatory.
\item
  To recognize the contributions of the individuals responsible for bringing the Rubin Project to fruition (including Rubin ``Builders'').
\item
  To enable publications by the broader community that use technical Rubin data or information not yet publicly available, or instrumentation developed by the Rubin Project.
\end{itemize}

\section{Authorizing Body}\label{authorizing-body}

The Rubin Project Science Team is the authorizing agent for the Rubin Project Publication Policy and must approve any future revisions.

\section{Publication Manager and Publication Board}\label{publication-manager-and-publication-board}

A Publication Manager is appointed for a renewable term of three years by the Rubin Observatory Director, in consultation with the Project Science Team.
The Publication Manager reports to the Project Science Team.
The Publication Manager oversees the logistics of the publication process, including the following responsibilities:

\begin{itemize}
\item
  working with the Project to maintain a public archive of Rubin Project publications;
\item
  overseeing the publication review process described in subsequent sections;
\item
  responding to requests by the community for use of Rubin intellectual property in publications;
\item
  assisting in the development of future publication policies; and
\item
  providing initial adjudication of publication-related disputes.
\end{itemize}

Quality control of the publication is the responsibility of the appointed review committee (see below), with oversight by the Publication Manager.

The Publication Manager chairs the Rubin Publication Board, which advises the Publication Manager on a range of issues such as assignment of reviewers, requests for Rubin intellectual property, requests from the science collaborations, conflict resolution, and open-source issues.
The Board membership is chosen to be broadly knowledgeable of Rubin hardware, software, and science requirements
Members of the Board are appointed for two-year terms by the Publication Manager in consultation with the Rubin Director and the Project Science Team.
During construction, the Rubin Director, Project Manager, Project Scientist, and Systems Scientist are active ex-officio members of the Publication Board.
After the transition to Operations, the ex-officio members of the Publication Board are the Head of LSST and the Associate Director for System Performance.
The roles and composition of the Project Science Team may also evolve in the transition to Operations.

The Publication Manager, in consultation with the Publication Board, is responsible for resolving conflicts that arise related to Rubin Project publications.
If the Publication Manager is unable to resolve publication-related disputes, the final authority rests with the Rubin
Director, advised by the Project Science Team.

\section{Publications Covered by this Policy}\label{publications-covered-by-this-policy}

This policy covers the public presentation of Rubin requirements and products by Observatory personnel\footnote{
  In this document, \emph{Observatory personnel} means anyone supported by the Observatory for the infrastructure or performance work reported in the paper, as well as anyone in an official Observatory position who could reasonably be construed to be representing the Observatory.
}.
Two types of publications are recognized in this policy: 1) technology papers that describe Rubin infrastructure and/or performance, and 2) data release papers.\footnote{
  After operations begin, data will be made available immediately to communities that need and can handle immediate unprocessed data (e.g., transient alerts).
  Rubin Observatory will make available annual releases of calibrated, documented data with tools to use it.
}
Rubin publications that describe the Rubin hardware and software infrastructure, subsystems, and simulations are
designated here as technology papers.
This designation is bestowed by the Rubin Director, with input from the Publication Manager, Management Team, and Project Science Team; the list of technology papers will be maintained by the Publication Manager on a public archive.

This policy does not cover Rubin papers that rely only upon public data.
Science collaborations are encouraged to develop publication policies that recognize the contributions of those who brought the Project to fruition, through citation of relevant technical and data release
publications.
Nor does this policy cover community-based publications that are considered \emph{inputs} to the Observatory, such as those exploring opportunities or optimization, even if those papers have Observatory personnel as co-authors.

This policy also does not generally cover posters, presentations (\emph{e.g.}, those given at talks) and abstracts, DocuShare documents handled by the CCB, tech notes\footnote{
  Observatory tech notes have their own approval rules concerning embargoed information.
}, and \url{https://community.lsst.org} forum posts.
Individual conference talks and posters fall under the Publication Policy only if the author is representing the Rubin Project, or the work was supported by Rubin Project funds and the presented work is based on technical Rubin information that is not yet public.

Journal papers and conference papers (\emph{e.g.}, AAS and SPIE meetings) fall under this policy, with some adjustments allowed for practical considerations.
See Section \ref{builders}.

The Publication Manager will be responsible for coordinating joint submissions for special journal issues and for major conferences where it is important that all submissions tell a consistent story.

We define two categories of papers:

\begin{itemize}
\item
  \textbf{Category 1} papers are major milestone papers that involve the integrated Observatory.
  Examples include the as-built system overview paper, the commissioning performance summary paper, and the major data release papers.
\item
  \textbf{Category 2} papers describe subsystems or specific details of some aspects of the Observatory.
  Examples include a paper on the Camera refrigeration system, or the suite of Data Management subsystem papers, or Calibration papers, or conference contributions.
\end{itemize}

The lead author will propose the category designation to the Publication Manager at the start of the process.
The different categories follow somewhat different authorship and review processes, as described in the following sections.

\section{\texorpdfstring{Authorship }{Authorship }}\label{authorship}

The authorship and citation policy for the Rubin Project follows common ethical practices as outlined by the American Astronomical Society (AAS) and the American Physical Society (APS):

\begin{itemize}
\item
  AAS Ethics Statement, Publication and Authorship Practices:
  \url{http://aas.org/about/policies/aas-ethics-statement}
\item
  APS Ethics and Values, Publication and Authorship Guidelines:
  \url{http://www.aps.org/policy/statements/02_2.cfm}
\end{itemize}

See Appendix \ref{appendix-authorship-policies-of-the-aas-and-aps} for the most relevant portions of each statement.

There are two ways in which those contributing to the Project can be acknowledged, depending on the significance of the contribution to the work described in the publication.

\begin{enumerate}
\def\labelenumi{\arabic{enumi}.}
\item
  When a publication describes a focused research effort that uses previously published work of the Project, the published Project work should be acknowledged by direct citation.
\item
  Authorship, as appropriate, on Project papers.
  In all cases, everyone who has made significant contributions to the paper will be invited and encouraged to be a co-author.
  The eligible author lists depend on the paper category:

  \begin{enumerate}
  \def\labelenumii{\alph{enumii}.}
  \item
    \textbf{Category 1 papers.} These major milestone papers are intended to have very broad authorship, including all Builders (see section \ref{builders} below) along with all Rubin Observatory team members who have made significant contributions to the Rubin Observatory, enabling the results reported in the paper (\emph{e.g.}, the Observing team, and others who might not yet have reached Builder status).

    \begin{enumerate}
    \def\labelenumiii{\roman{enumiii}.}
    \item
      The author list order for Category 1 papers are alphabetical by last name.
    \item
      Authorship sign-up is an online opt-in process that occurs after the internal review process is complete, so that authors know what they are signing.
      See section \ref{publication-review-and-notification-process} on Review and Notification process, below.
      Each author will check a box that says something like, ``I have read the paper and agree with its contents.
      I believe I have contributed significantly to enabling this paper.''
      The current opt-in list of authors is visible to the Rubin team and Builders throughout the sign-up process.
    \item
      A small number (1-3) authors will be designated as contact authors, with email addresses.
    \item
      The first author on Category 1 papers will be, ``The Rubin Observatory Team'', signifying that this paper represents the Observatory.
    \end{enumerate}
  \item
    \textbf{Category 2 papers.} The author list order for these papers is flexible, based on the specific situation and the consensus of the authors.
    Typically, the author list for these papers will be in two sections (or more, if there are external authors): the first section lists those who directly contributed to the paper or to specific work described in the paper; and the second section is an alphabetical listing of any Builders who are not already in the first section.

    \begin{enumerate}
    \def\labelenumiii{\roman{enumiii}.}
    \item
      The authorship opt-in process for Builders is like that of Category 1 papers, with one difference: while all Builders are strongly encouraged to sign Category 1 papers, for Category 2 papers the expectation is that Builders will sign only those papers relevant to their contributions to Rubin Observatory.
    \item
      The author list order in the first section is determined by agreement of the co-authors.
      If the author list order is purely alphabetical, a small number (1-3) authors will be designated as contact authors, with email addresses; also, where appropriate, and in consultation with the Publication Manager, a non-human first author for the team description (\emph{e.g.}, for a subsystem paper) is also possible.
      The author list order for the second section will be alphabetical.
    \item
      Each co-author is expected to review the manuscript before its submission and must give explicit permission to be listed as a co-author.
    \item
      Scientists and engineers, including Builders, may request being listed in the first section on Category 2 papers if they have made significant contributions to the paper or to the work described in the paper.
      These requests are submitted via a tracked online process (\emph{e.g.}, Google form or Jira\footnote{
        The Publication Board will maintain standard text for the opt-in form and the notification emails.
        The opt-in online process choice (Google form or Jira) is made by the lead author(s) for the paper, in consultation with the Publication Manager.
      }) to the lead author(s).
      The lead author is encouraged to consult with the Publication Manager or any Pub Board member about authorship questions.
      Any disagreements about authorship will be resolved as stated in Section \ref{publication-manager-and-publication-board}.
    \item
      Unlike Category 1 papers, Category 2 paper author lists will not include the designation, ``The Rubin Observatory Team''.
    \end{enumerate}
  \end{enumerate}
\end{enumerate}

Potential contributing authors should be engaged in the review of the content of the paper as early in the production of the paper as practical.

Rubin Project publications may include authors who are not members of the Rubin Project.

\section{Builders}\label{builders}

To ensure that proper credit is given to those responsible for the design, construction, and operation of the hardware, software, and other infrastructure of the Rubin Observatory, a list of ``Builders'' will be maintained by the Rubin Project Director (or designee) and posted to the Rubin Project website.
Builders are those who have made substantial contributions to the Rubin Project as a whole, including (but not limited to) the areas of optics, telescope, infrastructure, calibration, camera, simulations, data management, data pipeline software, commissioning, operations, and Project management.

To further ensure that the contributions of Rubin Builders are recognized, all science publications based on Rubin data should reference the appropriate technology papers --- \emph{i.e}., those Rubin publications that describe the Rubin infrastructure relevant to the study in question.
This implies, in general, that every science publication based on Rubin data must reference at least the primary Rubin technical overview paper that describes the data used in the analysis.
The individual Rubin science collaborations are encouraged to develop their own publication guidelines that ensure recognition of the contributions of Rubin Builders through references to technology and data release papers.
See the Rubin publication page\footnote{https://rubinobservatory.org/for-scientists/documentation} for key Rubin papers to cite in a publication\footnote{\url{https://github.com/lsst-pst/LSSTreferences}} and a template publication policy for Rubin science collaborations.

See the separate document \emph{Citation Needed} on Rubin Builders for details on eligibility for the Builders list and maintenance of the list.

\section{Publication Review and Notification Process}\label{publication-review-and-notification-process}

Internal paper review is a critical aspect of the Rubin Project publication process for technology and data release papers.
It is in the interest of the Rubin Project and the broader community to ensure that only reliable and accurate results are published and that all interested Rubin Project members have an opportunity to contribute to the publication through a constructive internal review process.
The benefit to the primary authors is the engagement of Rubin Project members with relevant expertise.

After operations begin, Rubin plans to make available annual public releases of calibrated, documented data with tools to use it.
It is likely that the Rubin Project will produce publications to describe each of these data releases.
As noted above, data release papers are Category 1, so the author list will be alphabetical.

We outline the review and notification process of Category 1 papers here:

\begin{enumerate}
\def\labelenumi{\arabic{enumi}.}
\item
  A Confluence page, or equivalent, maintained by the Publication Manager and Publication Board, will provide the list of planned Category 1 papers, each with lead authors or editors identified, along with a timeframe and set of projected milestones for each paper.
  When that page is updated with a new paper, Builders and Observatory team personnel will be notified.
\item
  When a first draft of the paper exists, a Jira or equivalent page is set up in the Publication Board system, with notification to the Project team and Builders email lists.
\item
  When the Publication Manager or designee agrees that the paper is ready for review by the Publication Board, t he Publication Manager (or designee), in consultation with the Publication Board, appoints a review committee for the paper consisting of typically three (at the Publication Manager\textquotesingle s discretion) active members of the Rubin Project or, if appropriate, science collaborations.
\item
  The review committee members will agree to submit their initial feedback to the authors and the Publication Manager at a set time, typically within two weeks.
\item
  The authors should explicitly respond to the feedback from the review committee and post an updated draft of the paper.
  Authors should consider acknowledging (in an Acknowledgement section of the paper) the contributions of any reviewer who has had a particularly significant impact on the content and quality of the paper.
\item
  Once the review committee members are satisfied, the Publication Manager posts the paper to an internal Rubin Project site (usually the Publication Board Jira page for the paper) for a two-week comment period, the author list opt-in site is opened, and the paper announcement is sent to the Project and Builders lists, with explicit deadlines.
\item
  The primary authors are responsible for responding to comments and feedback from Rubin Project members and Builders.
  The review committee members and the Publication Manager in consultation with the subsystem leads will arbitrate disputes on the scientific or technical content of the paper.
\item
  The Publication Manager is responsible for verifying that this process is followed and that all appropriate Rubin documents are cited.
\item
  The paper may not be submitted for publication without final approval by the review committee and the Publication Manager.
\item
  Once the paper has been approved for submission, the lead author is in   charge of the logistics of the publication and will submit the paper.
\item
  All referee reports, responses to the referee(s), and paper revisions will be made available to the Project Team and Builders on the internal paper site.
\end{enumerate}

The process for Category 2 papers is similar, except:

\begin{itemize}
\item
  The number of reviewers and the timeframe for the review may be different, to be determined by the Publication Manager in consultation with the Publication Board.
  For example, SPIE and similar papers might need only one internal reviewer.
\item
  The time for Observatory-wide comment may be as short as one week in some cases.
\item
  The open author sign-up process will include the definition of paper categories to assist Builders in their choice of whether or not to sign the paper (see Section \ref{authorship}).
\item
  For conference contributions (\emph{e.g.}, SPIE), Builders are not notified in the paper announcement to the Observatory team.
  Builders do see the page with all planned Category 2 papers, however.
\end{itemize}

To the extent possible, the authors will archive notes and software related to the published work in public or internal Rubin repositories or document systems.
These notes could include relevant details not included in the paper, descriptions and location of simulations and test data, relevant presentations, analysis code, communications between the analysis team and the review committee that are not already in an archived Rubin system, etc.
To the extent possible, the communication between the primary authors and the review committee, and other materials related to the work and the review, should be archived using tools created by the Project.
This material will provide an institutional memory of the work and will serve as an aid to future Rubin work.

Practical exceptions to this process are expected.
For example, if a contribution is severely limited in page length or is submitted as a conference paper, then a large number of authors, an extensive reference list, or a full acknowledgment may not be feasible.
The Publication Manager has the authority to grant partial exceptions in such cases.
When a large number of conference contributions are being prepared with a limited timeline for submission, shorter review periods with fewer internal reviewers may be necessary.
These modifications to the review process will be at the discretion of the Publication Manager, who will have the responsibility to anticipate major conferences for which such flexibility will be appropriate.

\subsection{Science Papers Based on Rubin
Products}\label{science-papers-based-on-rubin-products}

Before Rubin operations begin, we can expect to see scientific papers based on Rubin products such as the output of the operations simulator, the input catalog of stars and galaxies from which Rubin simulations are generated, the simulated images of stars and galaxies, or the output catalogs of the processed images.
Some of these papers will focus on Rubin performance and science reach; others will focus on topics that are not specific to Rubin.

If the Rubin product is public and described in a citable publication, the publication is referenced and no review is necessary.

If the product is public but no citable reference exists, or the product is not yet publicly available, then the following review process is recommended.
The benefit to the authors is the engagement of Rubin Project members with relevant expertise.

The lead author should contact the Publication Manager, who will identify the expert within the Rubin Project for the Rubin data product used in the paper.
The Publication Manager and the relevant expert will evaluate whether the proposed use of the product is appropriate.
If the analysis described in the paper makes a statement about Rubin performance, then the expert for the product (or his or her designee) will be given the right to review the paper to determine whether the data product was used appropriately.
Authors should consider acknowledging the contribution of any reviewer who has had a particularly significant impact on the content and quality of the paper.

\subsection{Acknowledgments}

Rubin Project publications will contain a standard text that acknowledges funding entities in the Rubin Project; authors may supplement this text with additional acknowledgement as required.
See \citeds{Document-3607} for details of the current acknowledgments text.
The Rubin Communications Manager is responsible for maintaining this material.

\section{Requests for Use of Rubin Intellectual
Property}\label{requests-for-use-of-rubin-intellectual-property}

On occasion there will be requests from parties external to the Rubin Observatory for use of Rubin-developed intellectual property (IP) for a variety of purposes other than those discussed above.
Examples would be for commercial use, general improvement of algorithms or hardware, or for use of still proprietary algorithms or hardware for other applications.
The conditions for use differ, depending on whether or not the IP is part of the Rubin open-source license.
Under the open-source open-data policy, reuse of any publicly released Rubin data or IP would fall under the open-use condition:

\begin{enumerate}
\def\labelenumi{\arabic{enumi}.}
\item
  Proper attribution of the source, and
\item
  if the IP is modified, the enhanced algorithm shall in turn be made public under the same open-source license.
\end{enumerate}

If neither of these conditions is satisfied, or if the request is for commercial use, individual agreements can be made with approval of the Rubin Director and the creators of the IP in question.
An example category would be use of Rubin images or other IP for limited commercial purposes, in which a one-time non-exclusive world license might be granted.
All such requests shall be handled on a case-by-case basis and referred to the Director for action.

\section{Press Releases}\label{press-releases}

All press releases that involve a publication that falls under the Rubin Project Publication Policy must follow the Rubin Press Release Policy \citedsp{Document-3604}.

\appendix

\section{Authorship policies of the AAS and
APS}\label{appendix-authorship-policies-of-the-aas-and-aps}

AAS Ethics Statement, Publication and Authorship Practices:
\url{http://aas.org/about/ethics_statement}
\emph{``All persons who have made significant contributions to a work intended for publication should be offered the opportunity to be listed as authors.
This includes all those who have contributed significantly to the inception, design, execution, or interpretation of the research to be reported.
People who have not contributed significantly should not be included as authors.
Other individuals who have contributed to a study should be appropriately acknowledged.
The sources of financial support for any project should be acknowledged/disclosed.
All collaborators share responsibility for any paper they co-author, and every co-author should have the opportunity to review a manuscript before its submission.
It is the responsibility of the first author to ensure these.''}

APS Ethics and Values, Publication and Authorship Guidelines:
\url{http://www.aps.org/policy/statements/02_2.cfm}
``\emph{Authorship should be limited to those who have made a significant contribution to the concept, design, execution or
interpretation of the research study.
All those who have made significant contributions should be offered the opportunity to be listed as authors.
Other individuals who have contributed to the study should be acknowledged, but not identified as authors.
The sources of financial support for the project should be disclosed.''}
